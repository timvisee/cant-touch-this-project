\documentclass[a4paper]{article}

\usepackage[dutch]{babel}
\usepackage{geometry}
\usepackage{inputenc}

\geometry{left=2.5cm, top=2.5cm, right=2.5cm, bottom=2.5cm}

\providecommand{\tightlist}{%
\setlength{\itemsep}{0pt}\setlength{\parskip}{0pt}}

\title{Touchless Interface Research}
\author{Tim Visée \& Nathan Bakhuijzen}

\begin{document}

\pagenumbering{gobble}

\maketitle

The Touchless Interface Research minor was supposed to be a project consisting
of three weeks of research, and three weeks of development. However, soon after
the start of the research we discovered that a different approach might be more
suitable for our project. In this paper we explain why the original approach was
not suitable enough, what our new proposal is, and why our proposal suits this
project better.

\subsection*{Original Approach}
The original approach defined in the slides of the minor describe a research
periode of threek weeks, followed by three weeks of development. This approach
does not fit well for our project, because our project is very experimental.
Using touchless devices to control a computer is a very general purpose, and
does not try to solve a particular problem. Therefore, it is very difficult to
pin down a certain subject to focus research on.

\subsection*{New Approach}
Our new approach is to develop a general purpose application that allows users
to control the computer using a touchless interface. In our project, the
LeapMotion device will act as our touchless interface. Before we start building
this application, we will look into a little bit general research about
controlling a computer using touchless interfaces. This will give us enough
information about the subject to point us in the right direction.

By developing a general purpose application we want to allow the researcher in
minor KB-81 to conduct experiments about the viability of touchless control of
a computer or other technical devices.

\subsection*{Arguments}
\begin{itemize}
  \tightlist
  \item Too broad
  \item Little on gestures
  \item Project in mind
  \item Why search for a problem?
  \item More effective: Research by trying
\end{itemize}

\end{document}
