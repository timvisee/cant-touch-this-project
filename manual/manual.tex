\documentclass[a4paper]{article}

\usepackage[dutch]{babel}
\usepackage{geometry}
\usepackage[utf8]{inputenc}
\usepackage{hyperref}
\usepackage{graphicx}

% dimensions
\geometry{left=3cm, top=3cm, right=3cm, bottom=3cm}

% font
\usepackage{DejaVuSans}
\renewcommand*\familydefault{\sfdefault}
\usepackage[T1]{fontenc}

% image
\graphicspath{ {img/} }

% lists
\providecommand{\tightlist}{%
\setlength{\itemsep}{0pt}\setlength{\parskip}{0pt}}

% preamble
\title{User Manual}
\author{Tim Visée \& Nathan Bakhuijzen}
\date{October 2018}

\begin{document}

  \pagenumbering{gobble}
  \maketitle
  \begin{figure}[h]
    \centering
    \includegraphics[width=\linewidth]{cant-touch-this}
  \end{figure}
  \clearpage

  \section{Problem Definition}
  \textit{What physical motions are natural, effortless and easy in order to
    control a computer or other digital device?}

  Questions of this nature can be answered using the \textit{Can't Touch This}
  platform. \textit{Can't Touch This} aim to be a general platform for
  researchers that would like to conduct research in this area. We believe that
  \textit{Can't Touch This} allows researchers to quickly identifty what motions
  are effective, and which aren't. Giving researchers the opportunity to conduct
  research improves the chance for touchless control of computers only seen
  in futuristic movies and tv shows.

  \subsection{Motivation}
  At the start of this minor, students were given a choice in the subject of the
  research. Mister Hani introduced us to a series of subjects, of which the
  LeapMotion project was the most interesting to us. The idea of the LeapMotion
  was to create or extend existing software to enable people to control a
  computer without touching any peripherals.

  \subsection{Background information}
  The events that lead up to this question came from the medical industry. When
  surgeons perform surgeries, they often have to look through medical files, for
  example, an X-ray scan. This is difficult, as the surgeons most likely has
  dirty hands, and can therefore not touch any computer peripheral.
  
  
  Instead of this, touchless control of the computer may offer a solution.

  The use of assistants is a solution to the problem, but this requires another
  human, and personnel in the medical world is scarce. We therefore want to
  replace these human assistants with a device that allows surgeons to control
  the computer without touch any peripherals.

  The people that are most interested in the results are software engineers,
  surgeons and hospital managers.
  \clearpage

  % Installation Guide
  %% Platform and OS
  %% Other software dependencies
  %% External resources needed to run the prototype
  %% External resources needed to develop the prototype
  \section{Installation Guide}
  \subsection{Requirements}
  \begin{itemize}
    \tightlist
    \item A computer with the Windows (7+), OSX (10.7+, Lion+) or
      Linux (kernel 2.6.18+) operating system
    \item An installation of the LeapMotion
      \href{https://developer.leapmotion.com/sdk/v2}{SDK}
    \item An installation of the
      \href{https://rust-lang.org}{Rust} programming language
    \item The physical LeapMotion device itself
  \end{itemize}

  \subsection{Software Dependencies}
  The \textit{Can't Touch This} platform is written using the
  \href{https://rust-lang.org}{Rust programming language}. This means that the
  operating system that the platform will run on must support the Rust language.
  Fortunately, Rust runs on all popular operating systems today, shown above in
  the list of requirements. An up-to-date list of all supported versions can be
  found on the
  \href{https://forge.rust-lang.org/platform-support.html}{Rust website}.
  Additonally, the \textit{Can't Touch This} platform requires the LeapMotion
  \href{https://developer.leapmotion.com/sdk/v2}{SDK} to provide all necessary
  sensor data. Just like the Rust programming language, the LeapMotion SDK can
  be installed on all platforms.

  \subsection{External resources}
  Additional software required to run the platform consists of the following
  items:

  Again, fortunately, these are all installed with the platform itself, and
  therefore requires no extra software.

  \subsection{External development tools}
  In order for other to continue developing the \textit{Can't Touch This}
  platform, they will need to install the following tools:
  \clearpage

  % User Instructions
  %% Instructions to run the experiment (explained in build plan)
  %% Aimed at non technical researchers
  %% Include screenshots when possible
  %% Bonus points: make a video of the instructions, upload to youtube and
  %% include link in manual

  % Requirements
  %% Here you describe what requirements you have already gathered from the client. 
  %% What requirements are done 
  %% What requirements are still open 
  %% Prioritizing of requirements (MOSCOW)

  % Architecture diagram
  %% Context view
  %% Functional view

  % Domain Model
  %% This is where you give future developers the information they need to
  %% continue the work. 
  %% In the domain model you list all: 
  %% API interfaces 
  %% Subsystems

  % Test Report
  %% Add this section even if you did not do any formal testing on the system.  
  %% What is the quality of the code 
  %% What has already been tested 
  %% What are the known bugs in the system

  % known issues
  \section{Known Issues}
  \begin{itemize}
    \tightlist
    \item \textit{Can't Touch This} may crash upon running the release version
      of the exectable
    \item On macOS, the LeapMotion device may never give data to begin with
    \item On macOS, the LeapMotion device may stop recording data randomly
    \item On macOS, the application may not run well when minimalizing the
      backend application
  \end{itemize}
\end{document}
