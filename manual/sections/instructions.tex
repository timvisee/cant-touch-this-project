\documentclass{standalone}
\usepackage{standalone}

\begin{document}
  \section{User Instructions}
  This chapter gives users instructions on how to use the \textit{Can't Touch
    This} platform. It assumes that the user has followed the instructions found
  in the \textit{Installation} chapter. The following instructions will detail
  how to setup the platform so that you can conduct the \textit{experiments}
  found further into this manual.
  % Step by step instructions on how to use the research platform to conduct
  % research.
  % Checklist:
  %   •	Instructions to run the experiment (explained in build plan)
  %   •	Aimed at non technical researchers
  %   •	Include screenshots when possible

  % Bonus points: make a video of the instructions, upload to youtube and
  % include link in manual
  % Note: even if video instructions are created the manual should still have
  % written instructions

  % TODO: Write detailed instructions on how to run the experiments

  \subsection{Usage}
  \begin{itemize}
    \tightlist{}
    \item Attach the LeapMotion device using it's provided USB cable
    \item Start the LeapMotion daemon application provided by the LeapMotion
      SDK, optionally use the provided LeapMotion control panel for this
    \item Start the \textit{Can't Touch This} platform by running the provided
      \emph{Can't Touch This} executable, or run it manually in a terminal from
      the project directory (\verb_cargo run_, or \verb_cargo run --release_ if
      supported). Rust nightly must be used (\verb_rustup default nightly_)
    \item Start up an web browser (Firefox, Chrome, Safari, etc) and navigate to
      \url{http://localhost:8000} on the same machine
    \item Click on '\textit{Start Recording}'
    \item Move your physical hand above the LeapMotion device to make a desired
      gesture
    \item Once you are done making the gesture, click on
      '\textit{Stop Recording}'
    \item The recorded gesture you've just made should be visibly represented on
      the canvas. Trim it to cut off undesired parts of the recorded gesture
    \item Name the new gesture template, and save the gesture by clicking on '\textit{Save Recording}'
    \item After recording, attempt to perform a recorded gesture above the
      sensor. The computer will give positive feedback by showing a notification
      in the web interface if the gesture is recognized
    \item The live visualizer can be toggled using the '\textit{Visualize}'
      button
  \end{itemize}

  \clearpage
\end{document}
