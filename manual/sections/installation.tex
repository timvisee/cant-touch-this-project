\documentclass{standalone}
\usepackage{standalone}

\begin{document}
  \section{Installation Guide}
  The \textit{Can't Touch This} platform requires the following:
  \begin{itemize}
    \tightlist{}
    \item A computer with the Windows (7+), OSX (10.7+, Lion+) or
      Linux (kernel 2.6.18+) operating system
    \item An installation of the LeapMotion
      \href{https://developer.leapmotion.com/sdk/v2}{SDK}
    \item An installation of the
      \href{https://rustup.rs}{Rust} programming language
    \item The physical LeapMotion device itself
    \item The \textit{Can't Touch This} platform
  \end{itemize}

  \subsection{Software Dependencies}
  The \textit{Can't Touch This} platform is written using the
  \href{https://rust-lang.org}{Rust programming language}. This means that the
  operating system that the platform will run on must also support the Rust
  programming language. Fortunately, Rust runs on all popular operating systems
  today, shown above in the list of requirements. An up-to-date list of all
  supported versions can be found on the
  \href{https://forge.rust-lang.org/platform-support.html}{Rust website}.
  Additonally, the \textit{Can't Touch This} platform requires the LeapMotion
  \href{https://developer.leapmotion.com/sdk/v2}{SDK} to provide all necessary
  sensor data. Just like the Rust programming language, the LeapMotion SDK can
  be installed on all platforms.

  \subsection{External resources}
  No additonal resources are required to run the \textit{Can't Touch This}
  platform.

  % rust nightly
  \subsection{External development tools}
  Continuing development of the \textit{Can't Touch This} platform requires
  basic tools like a text editor or IDE, and a terminal. It is highly
  recommended to use \href{https://git-scm.com/}{git}, as this was used during
  development of the platform. Additionally, setting up an CI server may prove
  useful. Setting up an CI server is beyond the scope of this manual.

  \clearpage
\end{document}
