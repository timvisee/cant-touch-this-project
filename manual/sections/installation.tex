\documentclass{standalone}
\usepackage{standalone}

\begin{document}
  \section{Installation Guide}
  The \textit{Can't Touch This} platform requires the following:
  \begin{itemize}
    \tightlist{}
    \item A computer with the Windows (7+), OSX (10.7+, Lion+) or
      Linux (kernel 2.6.18+) operating system
    \item The Rust programming language must be installed, recommended to
      install with \href{https://rustup.rs}{rustup}. Rust nightly must be used
      (\verb_rustup default nightly_)
    \item The \href{https://cmake.org/}{Cmake} toolchain must be installed
    \item The \href{https://freedesktop.org/wiki/Software/pkg-config}{pkg-config} tool must be installed
    \item The LeapMotion
      \href{https://developer.leapmotion.com/sdk/v2}{SDK} V2 must be installed
    \item The physical LeapMotion device itself
    \item The \textit{Can't Touch This} platform
      \href{https://gitlab.com/timvisee/cant-touch-this.git}{sources}
  \end{itemize}

  \subsection{Software Dependencies}
  The \textit{Can't Touch This} platform is written using the
  \href{https://rust-lang.org}{Rust} programming language. This means that the
  operating system that the platform will run on must also support the Rust
  programming language. Fortunately, Rust runs on all popular operating systems
  today, shown above in the list of requirements. An up-to-date list of all
  supported versions can be found on the Rust
  \href{https://forge.rust-lang.org/platform-support.html}{website}.
  Additonally, the \textit{Can't Touch This} platform requires the LeapMotion
  \href{https://developer.leapmotion.com/sdk/v2}{SDK} to provide all necessary
  sensor data. Just like the Rust programming language, the LeapMotion SDK can
  be installed on all platforms.

  \subsubsection{Install Rust}
  Installing the Rust programming language and it's toolchain is quite simple.
  The \href{https://rustup.rs}{rustup} tool can be used for this. The rustup
  website shows how the tool is installed on your platform.\\
  Check it out here: \url{https://rustup.rs}

  \paragraph{}
  To install rustup on Linux, first make sure you've \verb_curl_ installed.
  Then, initiate the installation by running the following command in a
  terminal:
  \lstinputlisting[
    language=sh,
    caption={%
      Install rustup and Rust
    },
  ]{listings/install-rust.sh}
  Follow the instructions shown by the installer, and make sure you select the
  \verb_nightly_ toolchain tuple. When installation is complete, you may check
  whether the Rust and carbo binaries are found by invoking:
  \lstinputlisting[
    language=sh,
    caption={%
      Check whether Rust and Cargo are installed properly
    },
  ]{listings/rust-version.sh}

  \paragraph{Nightly}
  Because this project currently requires a Rust nightly version due to the
  \verb_Rocket_ crate (library), it might break in future Rust versions. The
  last Rust version we've tried to be working successfully is
  \verb_2018-10-30_. Generally speaking, if the project compiles, it works. If
  it doesn't, switch to this version using the following commands:

  \lstinputlisting[
    language=sh,
    caption={%
      Switch to old Rust nightly
    },
  ]{listings/old-nightly.sh}

  You can always check what the last successful Continuous Integration build
  was \href{https://gitlab.com/timvisee/cant-touch-this/pipelines}{here},
  and use that Rust version.

  \subsubsection{Install Cmake and pkg-config}
  How to install \verb_Cmake_ and \verb_pkg-config_ falls outside the scope of
  this manual and varies greatly between platform and version.

  It's recommended to check out the respective websites for installation
  instructions.
  \begin{itemize}
    \tightlist{}
    \item \url{https://cmake.org/install/}
    \item \url{https://www.freedesktop.org/wiki/Software/pkg-config/}
  \end{itemize}

  \subsubsection{Install LeapMotion SDK V2}
  The LeapMotion SDK V2 can be downloaded from
  \href{https://developer.leapmotion.com/sdk/v2}{this} page. At the moment of
  writing this manual, an account is required thus you must register in order
  to to start the download. For developing this platform we've used version
  \verb_v2.3.1_, although newer versions having the major version V2 should work
  fine.

  \paragraph{Windows}
  On Windows, using the LeapMotion SDK installer obtained from the ZIP file
  provided on their website should be sufficient. Be sure to check the included
  \verb_README_ for an up-to-date set of installation instructions. If you
  want to be able to build the platform from source, make sure you also install
  the SDK files.

  \paragraph{Linux \& macOS}
  Installing the SDK on Linux or macOS requires a little bit more work.
  First install the LeapMotion daemon and control panel using the installer
  provided in the ZIP file from their website. Then the library files must be
  installed.

  Because some manual work is required for this, we've created a simple
  installation script (that is also used in the Continuous Integration
  environment), and we recommend to use this script. To do this, first clone our
  project repository. Then navigate into the \verb_ci_ directory, and run the
  \verb`install_sdk` script:
  \lstinputlisting[
    language=sh,
    caption={%
      Install the LeapMotion SDK library files with script
    },
  ]{listings/leap-motion-lib-auto-install.sh}
  You may also inspect this script to do each step manually. It can also be
  found \href{https://gitlab.com/timvisee/cant-touch-this/blob/ece65fab15b3c088e57628dc0c82474efddbbfd2/ci/install_sdk}{here}.

  \paragraph{}
  Alternatively you can manually install these files. For this, you must move the library
  files into the respective location on your machine.
  \lstinputlisting[
    language=sh,
    caption={%
      Copy and install LeapMotion library files
    },
  ]{listings/leap-motion-lib-manual-copy.sh}

  The installed \verb_pkg-config_ tool is used to tell the compiler where the
  library installed. For this, a \verb_pkg-config_ configuration file must be
  created. An example can be found in the project repository
  \href{https://gitlab.com/timvisee/cant-touch-this/blob/ece65fab15b3c088e57628dc0c82474efddbbfd2/ci/libleap.pc}{here}.
  Name it \verb_libleap.pc_ and place it at
  \verb_/usr/lib/pkgconfig/libleap.pc_. If you've used the \verb`install_sdk`
  script this is already done for you.

  To validate that the \verb_pkg-config_ configuration file was installed
  correctly, run the command \verb_pkg-config --libs libleap_. This should
  report \verb_-lLeap_.

  \paragraph{}
  Again, be sure to read the included \verb_README_ for up-to-date instructions.

  \subsection{External resources}
  No additonal resources are required to run the \textit{Can't Touch This}
  platform.

  % rust nightly
  \subsection{External development tools}
  Continuing development of the \textit{Can't Touch This} platform requires
  basic tools like a text editor or IDE, and a terminal. It is highly
  recommended to use \href{https://git-scm.com/}{git}, as this was used during
  development of the platform. Additionally, setting up an Continuous Integration server may prove
  useful. The current project repositories have a Continuous Integration
  environment configured for automated testing, but setting one up beyond the
  scope of this manual 

  \clearpage
\end{document}
