\documentclass{standalone}
\usepackage{standalone}

\begin{document}
  \section{Goal}
  The purpose of this manual is twofold:
  \begin{itemize}
    \tightlist{}
    \item To inform research how to use the platform to execute experiments and
      conduct research
    \item To inform future developers how to continue developing the platform
  \end{itemize}

  \subsection{Problem Definition}
  \textit{What physical motions are natural, effortless and easy, in order to
    control a computer or other digital device?}

  Questions of this nature can be answered using the \textit{Can't Touch This}
  platform. \textit{Can't Touch This} aims to be a platform for researchers that
  would like to conduct research in the field of touchless computer systems. We
  believe that our platform allows researchers to build a strong foundation for
  the future of touchless control. Giving researchers the opportunity to conduct
  research improves the chance for touchless control of computers only seen in
  futuristic movies and tv shows.

  \subsection{Motivation}
  At the start of the KB-80 minor, students were given a choice in the subject
  of the research. Mister Al-Ers introduced us to a series of subjects, of which
  the LeapMotion project was the most interesting. The idea behind the
  LeapMotion project was to create or extend existing software to enable people
  to control a computer without touching any peripherals, like keyboards and
  mice.

  \subsection{Background information}
  Research in the field of touchless computer systems is motivated by the desire
  for comparable systems in sterile environments. For example, surgeons often
  make use of computer systems to aid them during their surgeries by providing
  crucial information such as CT, MRI and X-ray scans. This is where touchless
  computer systems may offer a solution. Touchless computer systems allows
  surgeons to control a system without the need for physical peripherals.
  \clearpage
\end{document}
