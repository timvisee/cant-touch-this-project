\documentclass{standalone}
\usepackage{standalone}

\begin{document}
  \section{Requirements}
  This chapter details the list of open and finished requirements of the
  \textit{Can't Touch This} platform. The prioritization of this list is not
  according to the MoSCoW method, because we had to write the entire platform
  from scratch. This took a long time, with lots of challenges and unexpected
  issues. We therefore chose to figure out what to do on the fly, rather than
  planning things out beforehand.

  The following requirements have been satisfied:
  \begin{itemize}
    \tightlist{}
    \item Operating System independant
    \item Web interface for user interaction
    \item Predefined gestures
    \item Record new gestures
    \item Gestures management
    \item Multi-finger recognition
  \end{itemize}
  By using the Rust programming language we inherently met the operating system
  requirement. This saved us a lot of work and allowed us to focus on the
  platform's functionality. Similarly, we chose to use a web interface for the
  user interaction. These decisions saved us a substantial workload.

  The following requirements are left unsatisfied:
  \begin{itemize}
    \tightlist{}
    \item Gesture bound actions
    \item Combining multiple sensors
  \end{itemize}
  Gesture bound actions are not yet implemented, because we did not believe this
  is an important feature. We instead worked on gesture recognition and template
  management. Instead of binding actions to gestures, we give users visual
  feedback when the gestures is recognized.

  When we started out with the project, we wanted to see if it was possible to
  combine multiple LeapMotion sensors in order to achieve increased data
  accuracy. We looked into this, but we found that this was impossible, due to
  the proprietary SDK. Several websites pointed out that only the developers of
  the LeapMotion we able to do this.

  We contacted the LeapMotion team ourselves, but unfortunately they will not
  continue development on the LeapMotion. New and more interesting projects keep
  the team occupied, and the leapMotion itself is approaching obsolescence.
  \clearpage
\end{document}
