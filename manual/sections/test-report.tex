\documentclass{standalone}
\usepackage{standalone}

\begin{document}
  \section{Test Report}
  \subsection{Code Quality} % •	What is the quality of the code
  \subsection{Existing Tests} % •	What has already been tested
  The platform currently has a few unit tests that cover basic operations, such
  as the conversion of a \verb_Point_ to a \verb_PointTrace_, and more. We also
  have set up a few tests to cover the comparison of traces, such as straight
  lines and curves. Below you can see the a unit test for a straight line code:
  \lstinputlisting[
    language=Rust,
    caption={%
      Straight line unit test
    },
  ]{code/straight.rs}

  This unit test creates \verb_points_, a trace of \verb_Point3_'s, and compares
  it to \verb_expected_, a \verb_RotTrace_.

  The \verb_PointTrace_ on line 3 contains three points that travel the same
  distance at every step. The platform recognizes this as a straight line, as
  there is no change in the trajectory. The variable \verb_expected_ then gets
  assigned a \verb_RotTrace_ that contains only one \verb_RotPoint_ of 0
  degrees. This is correct, as the line drawn on line 3 is straight.

  \lstinputlisting[
    language=Rust,
    caption={%
      Corner unit test
    },
  ]{code/corner.rs}
  The \verb_corner_ test is a little more complicated than the \verb_straight_
  unit test. It creates a \verb_PointTrace_ with points that represent a 2D
  square. This motion first moves to a point, 5 units onto the \verb_y_ axis.
  After moving on this axis, it moves 5 units on the \verb_x_ axis. It then
  concludes the motion by moving back 5 units on the \verb_y_ and \verb_x_ axis,
  respectively.

  The \verb_RotTrace_ on line 11 contains three \verb_RotPoint_'s of \verb_-90_
  degrees. This is because a \verb_RotPoint_ calculates itself based on three
  \verb_Point_'s. In the case of this unit test, it takes three collections of
  three \verb_Points_: \verb_1-3_, \verb_2-4_ and \verb_3-5_. For each
  collection, it calculates 2 angles: \verb_1-2_ and \verb_2-3_. It then
  compares the difference between these angles, and that will be the
  \verb_RotPoint_ for those three points. This will produce three
  \verb_RotPoint_'s of \verb_-90_ degrees.

  \subsection{Known Bugs}
  \begin{itemize}
    \tightlist{}
    \item \textit{Can't Touch This} may crash upon running the release version
      of the exectable
    \item On macOS, the LeapMotion device may never give data to begin with
    \item On macOS, the LeapMotion device may stop recording data randomly
    \item On macOS, the application may not run well when minimalizing the
      backend application
  \end{itemize}
\end{document}
