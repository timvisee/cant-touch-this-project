\documentclass{standalone}
\usepackage{standalone}

\begin{document}
  \section{Test Report}
  \subsection{Code Quality}
  In order to keep quality of of the codebase in this project high, we've done
  the following things:

  \paragraph{}
  We've clearly defined a \verb_pipeline_, and modularized the project this way.
  Logic is localized to the corresponding module, and modules interconnect to
  pass data along as described in figure~\ref{fig:pipeline-diagram}.
  First of all, this makes the complete project much easier to
  understand as it's usually immediately visible in what part of the program
  something happens. Secondly, this makes working together on the project much
  better as developers can focus on the module they're working on without
  interfering with someone else's changes. Although unsure at this time,
  it probably makes modifying the code base to test various configurations for
  processing sensor data much better doable.

  \paragraph{}
  We use code linting to enforce a consistent code style across the whole
  project. This is done using \verb_rustfmt_, which is an additional tool that
  can be installed for Rust to format your whole project. This way developers
  work with a consistent code base.

  \paragraph{}
  Some unit tests have been implemented to ensure the most important parts of
  the project are working properly. Think of testing the logic to map a list of
  3D points into a trace of rotations, and a test to confirm resampling works
  properly. The unit tests itself are quite limited due to time constraints,
  this is explained in more details later on.

  \paragraph{}
  Our code base is hosted on GitLab, on which we've enabled Continuous
  Integration support. For each commit we push to the repository a server is
  automatically spun up to build and test (unit tests) the project to verify
  it is working properly. Along with testing, the Continuous Integration
  environment verifies that the code is properly formatted following the
  \verb_rustfmt_ guidelines, and if it isn\'t the \emph{build} fails.

  Along with this, the repository has been configured to block direct pushes to
  the \verb_master_ branch. Therefore a development branch must be created for
  each feature, after which a merge request can be created to merge new changes
  into the \verb_master_ branch, only if all tests through Continuous
  Integration succeed.

  \subsection{Existing Tests}
  The platform currently has a few unit tests that cover basic operations, such
  as the conversion of a \verb_Point_ to a \verb_PointTrace_, and more. We also
  have set up a few tests to cover the comparison of traces, such as straight
  lines and curves. Below you can see the a unit test for a straight line code:
  \lstinputlisting[
    language=Rust,
    caption={%
      Straight line unit test
    },
  ]{code/straight.rs}

  This unit test creates \verb_points_, a trace of \verb_Point3_'s, and compares
  it to \verb_expected_, a \verb_RotTrace_.

  The \verb_PointTrace_ on line 3 contains three points that travel the same
  distance at every step. The platform recognizes this as a straight line, as
  there is no change in the trajectory. The variable \verb_expected_ then gets
  assigned a \verb_RotTrace_ that contains only one \verb_RotPoint_ of 0
  degrees. This is correct, as the line drawn on line 3 is straight.

  \lstinputlisting[
    language=Rust,
    caption={%
      Corner unit test
    },
  ]{code/corner.rs}
  The \verb_corner_ test is a little more complicated than the \verb_straight_
  unit test. It creates a \verb_PointTrace_ with points that represent a 2D
  square. This motion first moves to a point, 5 units onto the \verb_y_ axis.
  After moving on this axis, it moves 5 units on the \verb_x_ axis. It then
  concludes the motion by moving back 5 units on the \verb_y_ and \verb_x_ axis,
  respectively.

  The \verb_RotTrace_ on line 11 contains three \verb_RotPoint_'s of \verb_-90_
  degrees. This is because a \verb_RotPoint_ calculates itself based on three
  \verb_Point_'s. In the case of this unit test, it takes three collections of
  three \verb_Points_: \verb_1-3_, \verb_2-4_ and \verb_3-5_. For each
  collection, it calculates 2 angles: \verb_1-2_ and \verb_2-3_. It then
  compares the difference between these angles, and that will be the
  \verb_RotPoint_ for those three points. This will produce three
  \verb_RotPoint_'s of \verb_-90_ degrees.

  \paragraph{}
  Note that the project only includes unit tests for a few essential and easily
  testable parts. Due to time constraints we didn't have time left to implement
  more and more comprehensive tests.

  \subsection{Known Bugs}
  \begin{itemize}
    \tightlist{}
    \item At the moment of writing, there seems to be an issue in the latest
      few Rust nightly builds which might cause the \textit{Can't Touch This}
      platform to crash when running in \verb_--release_ mode.
    \item On macOS, there seem to be yet unexplainable problems causing the sensor
      to stop reporting new data to our platform. We are unsure whether this is
      caused because of a bad machine installation, or because of macOS
      throttling unfocussed applications. A machine restart usually temporarily
      solves the issue.
  \end{itemize}
\end{document}
