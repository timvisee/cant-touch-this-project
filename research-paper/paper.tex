\documentclass[a4paper]{article}

\usepackage[dutch]{babel}
\usepackage{geometry}
\usepackage[utf8]{inputenc}
\usepackage{graphicx}

% Font
\usepackage{DejaVuSans}
\renewcommand*\familydefault{\sfdefault}
\usepackage[T1]{fontenc}

\graphicspath{ {img/} }

\geometry{left=3cm, top=3cm, right=3cm, bottom=3cm}

\providecommand{\tightlist}{%
\setlength{\itemsep}{0pt}\setlength{\parskip}{0pt}}


\title{Can't Touch This}
%\subtitle{An attempt at touchless control of a computer}
\author{Tim Visée \& Nathan Bakhuijzen}
\date{September 2018}

\begin{document}

  \pagenumbering{gobble}
  \maketitle
  \begin{figure}[h]
    \centering
    \includegraphics[width=\linewidth]{cant-touch-this}
  \end{figure}

  \clearpage

  \section{Problem Definition}
  Today's control of a computer is exclusively done through physical interfaces,
  whether this is through keyboards or other peripherals.

  \subsection{Motivation}
  At the start of this minor, students were given a choice in the subject of the
  research. Mister Hani introduced us to a series of subjects, of which the
  LeapMotion project was the most interesting to us. The idea of the LeapMotion
  was to create or extend existing software to enable people to control a
  computer without touching any peripherals.

  \subsection{Background information}
  The events that lead up to this question came from the medical industry. When
  surgeons perform surgeries, they often have to look through medical files, for
  example, an X-ray scan. This is difficult, as the surgeons most likely has
  dirty hands, and can therefore not touch any peripheral. Instead of viewing
  the documents himself, the surgeon has assistants that help him instead.

  The use of assistants is a solution to the problem, but this requires another
  human, and personnel in the medical world is scarce. We therefore want to
  replace these human assistants with a device that allows surgeons to control
  the computer without touch any peripherals.

  The people that are most interested in the results are software engineers,
  surgeons and hospital managers.
  \clearpage

  \section{Goal of research}
  The problem owner wants us to develop a platform where users can record their
  own gestures and bind these to general computer actions. This allows other
  students to do more extensive and complex experiments.
  \clearpage

  \section{Research Questions}
  \subsection{Main question}
  \textbf{Is it viable to control a computer with a touchless, gesture based
    interface?}

  In order to fully understand this question, we first need to define
  some terms:
  \begin{itemize}
    \tightlist
    \item Viable
    \item Control
    \item Interface
  \end{itemize}

  We define \textbf{viable} as: More \textit{efficient} than existing options or
  useful in specific environments.

  By \textbf{efficient} we mean:
  \begin{itemize}
    \tightlist
    \item \textit{Faster} in use
    \item \textit{Easier} in use
    \item \textit{Reliable} (detection rate)
    \item (less) \textit{Effort}
  \end{itemize}

  %\subsubsection{Subquestion 1}
  %\subsubsection{Subquestion 2}
  %\subsubsection{Subquestion 3}
  \clearpage

  \section{Nature of research}
  % Explorative, descriptive or hypothesis testing
  % Qualitative or quantitative
  Our research is of an explorative, qualitative nature. This is because we
  have developed a platform where researchers can experiment with touchless
  control of a computer. The experiments that the researchers can perform have
  not been performed yet, so this means the research is exploratory.
  \clearpage

  \section{Experiment setup}
  % Example of an experiment that students of minor KB-81 can perform.
  \clearpage

  \section{Research platform}
  % How will we develop the platform during the second half of the minor?
  % What will the platform do?
  % How will it work?
  \subsection{Description}
  \subsection{List of requirements (MoSCoW)}
  \clearpage

  \section{Appendix A.1} % Literature survey (8 relevant articles)

  \subsection{Article 1 title}
  \subsubsection{Article 1 goal}
  \subsubsection{Article 1 method}
  \subsubsection{Article 1 conclusion}
  \subsubsection{Article 1 relevance}
  \clearpage
  
  % Article 2..n
  %\clearpage

  \section{Appendix A.2} % Conclusion of all literature surveys
  \subsection{Reference 1}
  \subsection{Reference 2}
  \clearpage

\end{document}
