\documentclass[a4paper]{article}
\usepackage[dutch]{babel}
\usepackage{geometry}
\usepackage[utf8]{inputenc}
\usepackage{hyperref}
\usepackage{listings}
\usepackage{graphicx}
\usepackage[x11names, rgb, html]{xcolor}

% dimensions
\geometry{left=3cm, top=3cm, right=3cm, bottom=3cm}

% font
\usepackage{DejaVuSans}
\renewcommand*\familydefault{\sfdefault}
\usepackage[T1]{fontenc}

% hyperlinks
\hypersetup{%
  colorlinks=true,
  linkcolor=blue,
  urlcolor=cyan,
}

% image
\graphicspath{{img/}}

% lists
\providecommand{\tightlist}{%
\setlength{\itemsep}{0pt}\setlength{\parskip}{0pt}}

% preamble
\title{Experiments}
\author{Tim Visée \& Nathan Bakhuijzen}
\date{October 2018}

\begin{document}

  \pagenumbering{gobble}
  \maketitle
  \begin{figure}[h]
    \centering
    \includegraphics[width=\linewidth]{cant-touch-this}
  \end{figure}
  \clearpage

  \section*{Recognized motions}
  In this document, we will list all the motions that are recognized by
  \textit{Can't Touch This}. These experiments can be conducted by first setting
  up the platform itself. Instructions for this can be found in the manual.
  After setting everything up, conducting the experiments is a piece of cake.
  Start up the platform, open a webbrowser and go to \href{http://localhost:8000}.
  Click on `Visualize' to view your fingers being traced.

  \paragraph{}
  Please try to make the following gestures above the LeapMotion sensor. You
  will receive visual feedback if a gesture has been made succesfully.

  \begin{itemize}
    \tightlist
    \item Straight line
    \item Circle clockwise
    \item Circle counter-clockwise
    \item Big circle clockwise
    \item Big circle counter-clockwise
    \item Triangle clockwise
    \item Triangle counter-clockwise
    \item Mini square clockwise
    \item Square clockwise
    \item Square counter-clockwise
  \end{itemize}


  \section*{Recognition consistency}
  It is important that all gestures are succesfully performed and recognized
  multiple times. This is because recognizing gestures can be error-prone. By
  performing gestures often, the risk of it recognizing a gesture incorrectly
  decreases dramatically.

  \section*{Detection speed}
  Another important aspect of the gesture recognition system is the speed, or
  rather, the delay for detecting a performed gesture. To avoid confusion, we've
  decided to define the term delay as follows: \textit{The time it takes for the
    platform to show the name of the correct gesture in the console, after a
    gesture has been completely performed above the sensor}.

  \section*{Buffering}
  Since it can be tricky to accurately measure the time difference between an
  invocation and an action, we've decided to record 20 invocations. With this as
  buffer, we can count a delay in frames, with precision accurate enough to
  recognize gestures correctly. The square gesture uses this technique the most,
  because it has very distinctive corners.

  \section*{Internal testing results}
  After performing the our own tests it was quite clear that delay varies
  greatly. The \emph{delay} varied from about 5 to 11 frames. With a speed of 60
  frames per second, each frame is about 17 milliseconds. This defines a delay
  varying between 85 to 187 milliseconds. There were two outliers with a delay
  of 16 to 22 frames which translate to about 270 and 370 milliseconds.

  We believe the varying difference has to do with the accuracy of detecting the
  gesture itself. Imagine you are drawing a perfect square gesture (perfect
  relative to the square gesture you might have previously recorded) above the
  sensor, the gesture will be recognized quicker before even fully completing
  the gesture, as compared to a square motion that differs slightly. Because
  it's virtually impossible to perform the gesture in exactly the same way every
  iteration, a difference in detection is indeed expected.

  Note that our method of testing latency is still quite simplistic. It doesn't
  take computer and screen delay into account. Nor does it test what the latency
  of the used sensor itself is. Our test strictly focusses on the delay between
  performing the gesture and seeing visual feedback. We performed the test with
  the visualizer enabled, which was visible on the material we recorded. What's
  interesting is that it looks like the gesture is instantly (meaning; within
  the same frame) recognized as the last sampled point of a gesture shows up on
  the visualizer. This would suggest that the actual latency for detection on
  the processed data is faster than 17 milliseconds. This is of course not very
  scientific, but we feel it's an awesome result non the less.

  What is important though, is how responsive the detection feels. During these
  tests, the detection (strictly speaking about detection speed) felt snappy even
  during the case of those outliers. The cool thing is that after you've
  performed a gesture you don't have to wait for a detection notification before
  the next gesture can be performed, that helps quite a lot for it to feel
  responsive to our opinion when when repeatedly experimenting with the sensor.

  The platform doesn't support binding
  generic actions to gestures to control a computer yet. And until that is
  tested, it's hard to say whether we'd feel the same when controlling a
  computer with these gestures.

\end{document}
