\documentclass[a4paper]{article}

\usepackage[dutch]{babel}
\usepackage{geometry}
\usepackage[utf8]{inputenc}

% Font
\usepackage{DejaVuSans}
\renewcommand*\familydefault{\sfdefault}
\usepackage[T1]{fontenc}

\geometry{left=2.5cm, top=2.5cm, right=2.5cm, bottom=2.5cm}

\providecommand{\tightlist}{%
\setlength{\itemsep}{0pt}\setlength{\parskip}{0pt}}

\title{Touchless Interface Research}
\author{Tim Visée \& Nathan Bakhuijzen}

\begin{document}

\pagenumbering{gobble}

\maketitle

The Touchless Interface Research minor was supposed to be a project consisting
of three weeks of research, and three weeks of development. However, soon after
the start of the research we discovered that a different approach might be more
suitable for our project. In this paper we explain why the original approach was
not suitable enough, what our new proposal is, and why our proposal suits this
project better.

\subsection*{Original Approach}
The original approach defined in the slides of the minor describe a research
periode of three weeks, followed by three weeks of development.
The main idea was to research and build a platform focussed on the medical
sector, think of allowing touchless interaction with a machine to keep is
sterile.

After some research we quickly discovered there aren't much research
papers readily available that cover this subject. Based on this, we decided to
start working on a general purpose implementation first. This platform would
allow you to interact with a computer. Sadly, there aren't much research reports
available either that are relevant for our implementation. In addition, the
project doesn't solve a particular problem yet. Therefore, it is very difficult
to pin down a certain subject to focus research on. Because of this, we
decided to conduct our own research by experimenting on our own implementation.

\subsection*{New Approach}
As mentioned, our new approach is to develop a general purpose application that
allows users to control the computer using a touchless interface.
In our project, the Leap Motion device will act as our hand motion and gesture
sensor being interface to the computer.

Before we start building this application, we will look into a little bit
general research about the following topics:
\begin{itemize}
  \tightlist{}
  \item Controlling a computer with touchless interaction
  \item How viable and efficient is interaction with hand gestures versus a
      mouse and keyboard.
  \item Recognizing hand gestures based on coordinate data
  \item Combining sensors to improve data accuracy
\end{itemize}

We expect that the results will give us more than enough information about the
subject to point us in the right direction for a initial proper implementation.
Note that most of the research on the mentioned topics has already been
conducted at the time of writing.

The last few days of this minor, we will be be conducting some experiments.
We will define the research questions to do the experiments for in advance.
Our main goal of these experiments is to figure out whether certain hand
movements and gestures are usable, and whether it's viable to do general tasks
on a computer with it as compared to a keyboard and mouse.
The actual experiments will be better defined in the other documents we will
produce.

By developing a general purpose application we want to allow the researcher in
minor KB-81 to conduct experiments about the viability of touchless control of
a computer or other technical devices, and/or to develop the platform further.

\end{document}
